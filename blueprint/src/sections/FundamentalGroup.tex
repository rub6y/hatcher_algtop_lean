\section{Fundamental Group properties}

    \begin{definition}[$\beta_h$ map]
        \label{def:beta_h_map}
        \uses{def:fundamental_group}

        Given a path $h$ from $x_0$ to $x_1$ we definie a map $\beta_h : \pi_1(X,x_0) \to \pi_1(X,x_1)$ as
        $$\beta_h ([\gamma]) = [h \cdot \gamma \cdot \bar{h}]$$
    \end{definition}

    \begin{lemma}[$\beta_h$ map is well definied]
        \label{lem:beta_h_map_welldef}
        \uses{def:beta_h_map}

        Map $\beta_h$ is well definied.
    \end{lemma}

    \begin{proof}
        \href{https://pi.math.cornell.edu/~hatcher/AT/AT.pdf#page=37}{See here!}
    \end{proof}

    \begin{lemma}[$\beta_h$ map is isomorphism of groups]
        \label{lem:beta_h_map_iso}
        \uses{def:beta_h_map}

        Map $\beta_h$ is an isomorphism.
    \end{lemma}

    \begin{proof}
        \href{https://pi.math.cornell.edu/~hatcher/AT/AT.pdf#page=37}{See here!}
    \end{proof}

    \begin{lemma}[Fundamental Group doesnt depend on base point]
        \label{fundamental_group_basepoint_dependence}
        \uses{lem:beta_h_map_iso}

        For path connected space fundamental groups based on $x_0, x_1$ are isomorphic.
    \end{lemma}

    \begin{proof}
        \href{https://pi.math.cornell.edu/~hatcher/AT/AT.pdf#page=37}{See here!}
    \end{proof}

    \begin{definition}[Loops in $S^1$]
        \label{def:wn_loop}
        For $n \in \mathbb{Z}$ let us define:
        $$\omega_n(s) = (\cos{2\pi ns}, \sin{2 \pi ns})$$

        Note: $\omega_n$ is the loop running $n$-times around the circle clockwise or counterclockwise
        depending on the sign of $n$.
    \end{definition}

    \begin{lemma}[$\omega_n$ is a loop]
        \label{lem:wn_loop_is_loop}
        \uses{def:wn_loop, def:loop}

        For each $n \in \mathbb{Z}$: $\omega_n$ is a loop based in $(1,0)$.
    \end{lemma}

    \begin{definition}[Evenly covered]
        \label{def:evenly_covered}

        Let $f : X \to Y$ be a map and $U \subset Y$ be an open set. We say that $U$ is evenly covered
        by $f$ when $f^{-1}(U)$ is a union of disjoint open sets each of which is mapped homeomorphically onto $U$ by $f$.
    \end{definition}

    \begin{definition}[Covering space]
        \label{def:covering_space}
        \uses{def:evenly_covered}

        Given a space $X$, a covering space is a space $\tilde{X}$ together with a map $p : \tilde{X} \to X$ such that:
        For each point $x \in X$, $x$ has a evenly covered neighbourhood $U$ by $p$.
    \end{definition}

    \begin{lemma}[Homotopy lifting property]
        \label{lem:homotop_lifting}
        \uses{def:covering_space}

        Given a map $F:Y\times I \to X$ and a map $\tilde{F}: Y \times \{0\} \to \tilde{X}$ lifting $F|Y\times \{0\}$, then
        there is a unique map $\tilde{F}: Y \times I \to \tilde{X}$ lifting $F$ and restricting to the given $\tilde{F}$ on $Y \times \{0\}$.
    \end{lemma}

    \begin{proof}
        \href{https://pi.math.cornell.edu/~hatcher/AT/AT.pdf#page=39}{See here!} (Part "c)" in this proof ) \\
        \href{https://agorism.dev/book/math/top/intro-topological-manifolds_john-lee.pdf#page=240}{Or here!} Here it is done for $S^1$ but it can be swaped to $X$
    \end{proof}

    \begin{lemma}[Path lifting property]
        \label{lem:path_lifting}
        \uses{lem:homotop_lifting}

        For each path $f:I \to X$ starting at a point $x_0 \in X$ and each $\tilde{x_0} \in p^{-1}(x_0)$ there
        is a unique lift $\tilde{f}:I \to \tilde{X}$ starting at $\tilde{x_0}$.

    \end{lemma}

    \begin{proof}
            \href{https://pi.math.cornell.edu/~hatcher/AT/AT.pdf#page=39}{See here!}
    \end{proof}

    \begin{lemma}[Homotopy path lifting property]
        \label{lem:homotop_path_lifting}
        \uses{lem:homotop_lifting}

        For each homotopy $H : X \times I \to X$ of paths starting at $x_0$ and each $\tilde{x_0} \in p^{-1}(x_0)$ there
        is a unique lifted homotopy $\tilde{H} : X \times I \to \tilde{X}$ of paths starting at $\tilde{x_0}$.
    \end{lemma}

    \begin{proof}
            \href{https://pi.math.cornell.edu/~hatcher/AT/AT.pdf#page=39}{See here!}
    \end{proof}

    \begin{lemma}[$\omega_n$ are not homotopic]
        \label{lem:wn_diff_homoclass}
        \uses{lem:homotop_path_lifting, lem:path_lifting, lem:Rn_path_equiv_lemma, lem:wn_loop_is_loop}

        Let $n \neq m$ be integers. Then $\omega_n \nsim \omega_m$.
    \end{lemma}

    \begin{proof}
            \href{https://pi.math.cornell.edu/~hatcher/AT/AT.pdf#page=39}{See here!}
    \end{proof}

    \begin{lemma}[All $S^1$ loops are $\omega_n$]
        \label{lem:loop_S1_homotopic_wn}
        \uses{lem:homotop_path_lifting, lem:path_lifting, lem:Rn_path_equiv_lemma, lem:wn_loop_is_loop}

        Let $\gamma$ be a loop in $S^1$ based at $(1,0)$. Then there exists $n \in \mathbb{Z}$
        such that $\gamma \sim \omega_n$.
    \end{lemma}

    \begin{proof}
            \href{https://pi.math.cornell.edu/~hatcher/AT/AT.pdf#page=39}{See here!}
    \end{proof}

    \begin{lemma}[Additive structure on $\omega_n$]
        \label{lem:wn_comp_is_sum}

        Let $n,m \in \mathbb{Z}$. Then
        $$[\omega_n] \cdot [\omega_m] = [\omega_{n+m}]$$
    \end{lemma}

    \begin{theorem}[Fundamental group of S1]
        \label{thm:fundamental_group_of_S1}
        \uses{def:fundamental_group,lem:loop_S1_homotopic_wn, lem:wn_diff_homoclass, lem:wn_comp_is_sum}

        $\pi_1(S^1) \cong \mathbb{Z}$
    \end{theorem}

    \begin{proof}
            \href{https://pi.math.cornell.edu/~hatcher/AT/AT.pdf#page=38}{See here!}
    \end{proof}
