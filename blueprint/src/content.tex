% In this file you should put the actual content of the blueprint.
% It will be used both by the web and the print version.
% It should *not* include the \begin{document}
%
% If you want to split the blueprint content into several files then
% the current file can be a simple sequence of \input. Otherwise It
% can start with a \section or \chapter for instance.


\chapter{Fundamental Group of Circle}

    \section{Homotopy Definitions}

    In this section we provide all the definition , lemmas and theorems regarding homotopies. 
    At the end we provide the definition of Fundamental Group of Topological Space and proof that it
    has a group structure.

    \begin{definition}[Homotopy of maps]
        \label{def:homotopy}
        
        Let $X,Y$ be topological spaces. We say that maps $f,g : X \to Y$ are homotopic ($f \simeq g$) iff there exists a continoues map $H : X \times I \to Y$ such that
        for any $x \in X$ 
            $$H(x,0) =  f(x) \text{ and } H(x,1) = g(x)$$.
    \end{definition}

    \begin{definition}[Path]
        \label{def:path}

        A path between points $x,y \in X$ is a continoues function $\gamma : I \to X$ such that 
            $$\gamma(0) = x \text{ and } \gamma(1) = y$$

    \end{definition}

    \begin{definition}[Loop]
        \label{def:loop}
        \uses{def:path}

        A loop is a path where $x = y$.
    \end{definition}

    \begin{definition}[Homotopy of Paths]
        \label{def:path_homotopy}
        \uses{def:path, def:homotopy}

        We say that two paths $\gamma_1, \gamma_2$ ($\gamma_1 \simeq_p \gamma_2$) from $x$ to $y$ are homotopic iff there exists homotopy map $H : I \times I \to X$
        such that H is homotopy of $\gamma_1, \gamma_2$ and for all $t \in I$ function $H(\cdot, t)$ is a path from $x$ to $y$.
    \end{definition}

    \begin{lemma}[All paths from $x$ to $y$ in $\mathbb{R}^n$ are Homotopic]
        \label{lem:Rn_path_equiv_lemma}
        \uses{def:path_homotopy}

        Any two paths $\gamma_1, \gamma_2$ from $x$ to $y$ in $\mathbb{R}^n$ are homotopic.
    \end{lemma}

    \begin{theorem}[Homotopy is equvialence relation]
        \label{thm:homotopy_equiv}
        \uses{def:homotopy}

        Relation $\simeq$ is an equvialence relation.
    \end{theorem}

    \begin{theorem}[Homotopy of paths is equvialence relation]
        \label{thm:path_homotopy_equiv}
        \uses{thm:homotopy_equiv, def:path_homotopy}

        Relation $\simeq_p$ of paths is an equvialence relation.
    \end{theorem}

    \begin{definition}[Composition of paths]
        \label{def:path_composition}
        \uses{def:path}

        Given to paths $\gamma_1, \gamma_2$ we definte $\gamma_1 \cdot \gamma_2$ by the formula:

        $$
            \gamma_1 \cdot \gamma_2 (t) =
            \begin{cases}
            \gamma_1(2s), & \text{if } t \leq \frac{1}{2}, \\
            \gamma_2(1-2s), & \text{if } t \geq \frac{1}{2}
            \end{cases}
        $$
    \end{definition}

    \begin{definition}[Inverse of paths]
        \label{def:path_inverse}
        \uses{def:path}

        Given to paths $\gamma$ we definte $\bar{\gamma}$ (inverse of $\gamma$) by the formula:

        $$
            \bar{\gamma}(t) = \gamma(1-t)
        $$
    \end{definition}

    \begin{lemma}[Composition of paths is a path]
        \label{lem:path_comp_path}
        \uses{def:path,def:path_composition}

        Composition of paths is a path (The map given by \ref{def:path_composition} is continoues) 
    \end{lemma}

    \begin{lemma}[Inverse of paths is a path]
        \label{lem:path_inverse_path}
        \uses{def:path,def:path_inverse}

        Inverse of path is a path (The map given by \ref{def:path_inverse} is continoues) 
    \end{lemma}

    \begin{lemma}[Composition of paths depend on homotopy class]
        \label{lem:path_comp_homoclass}
        \uses{def:path, def:path_composition, def:path_homotopy,thm:path_homotopy_equiv}

        If $f_0 \simeq_p f_1$ and $g_0 \simeq_p g_1$ then $f_0 \cdot g_0 \simeq_path f_1 \cdot g_1$ 
    \end{lemma}

    \begin{lemma}[Inverse of paths depend on homotopy class]
        \label{lem:path_inverse_homoclass}
        \uses{def:path, def:path_inverse, def:path_homotopy,thm:path_homotopy_equiv}

        If $f_0,f_1$ are to homotopic paths then $\bar{f_0}, \bar{f_1}$ are also homotopic.
    \end{lemma}

    \begin{theorem}[Homotopy of loops is equvialence relation]
        \label{thm:loop_homotopy_equiv}
        \uses{thm:path_homotopy_equiv, def:path_homotopy}

        Relation $\simeq_l$ of loops is an equvialence relation. (We use $\simeq$ to simplify notation)
    \end{theorem}

    \begin{lemma}[Composition of loops is a loop]
        \label{lem:loop_comp_loop}
        \uses{def:loop, lem:path_comp_path}
        
        Composition of loops is a loop.
    \end{lemma}

    \begin{lemma}[Inverse of loop is a loop]
        \label{lem:loop_inverse_loop}
        \uses{def:loop,def:path_inverse_path}

        Inverse of loop is a loop (The map given by \ref{def:path_inverse} is continoues) 
    \end{lemma}

    \begin{lemma}[Composition of loops depend on homotopy class]
        \label{lem:loop_comp_homoclass}
        \uses{lem:loop_comp_loop, lem:path_comp_homoclass, thm:loop_homotopy_equiv}

        If $f_0 \simeq_p f_1$ and $g_0 \simeq_p g_1$ then $f_0 \cdot g_0 \simeq_p f_1 \cdot g_1$ 
    \end{lemma}

    \begin{lemma}[Inverse of loops depend on homotopy class]
        \label{lem:loop_inverse_homoclass}
        \uses{lem:loop_inverse_loop, lem:path_inverse_homoclass, thm:loop_homotopy_equiv}

        If $f_0,f_1$ are to homotopic loops then $\bar{f_0}, \bar{f_1}$ are also homotopic.
    \end{lemma}

    \begin{definition}[Fundamental Group]
        \label{def:fundamental_group}
        \uses{lem:loop_comp_homoclass, lem:loop_inverse_homoclass}

        We definie the fundamental group of $(\pi_1(X, x_0),\cdot)$ as the set of equvialence classes of relation $\simeq$ with 
        the operation $\cdot$ - composition of loops
    \end{definition}

    \begin{lemma}[Composition is associative]
        \label{lem:loop_comp_assoc}
        \uses{def:fundamental_group}

        The operation $\cdot$ is associative.
    \end{lemma}

    \begin{lemma}[Composition has natural element]
        \label{lem:loop_comp_neutral}
        \uses{def:fundamental_group}
        
        There is an neutral element of $\cdot$, which is $[\text{const}_{x_0}]_{\simeq}$
    \end{lemma}

    \begin{lemma}[Composition has inverse]
        \label{lem:loop_comp_inv}
        \uses{def:fundamental_group}

        For every element of $\pi_1(X, x_0)$ there exists an inverse such that: 

        $[f] \cdot [g] = [\text{const}_{x_0}]$
    \end{lemma}
    
    \begin{theorem} [Fundamental Group is a Group]
        \label{thm:fundamental_group_is_group}
        \uses{lem:loop_comp_inv, lem:loop_comp_neutral, lem:loop_comp_assoc}

        The fundamental group is a group
    \end{theorem}

    \begin{theorem} [Fundamental Group of $\mathbb{R}^n$]
        \label{thm:Rn_fundamental_group}
        \uses{thm:fundamental_group_is_group, lem:Rn_path_equiv_lemma}
        
        The fundamental group of $\mathbb{R}^n$ is trivial

    \end{theorem}

    

    \section{Fundamental Group properties}

    \begin{definition}[$\beta_h$ map]
        \label{def:beta_h_map}
        \uses{def:fundamental_group}

        Given a path $h$ from $x_0$ to $x_1$ we definie a map $\beta_h : \pi_1(X,x_0) \to \pi_1(X,x_1)$ as
        $$\beta_h ([\gamma]) = [h \cdot \gamma \cdot \bar{h}]$$
    \end{definition}

    \begin{lemma}[$\beta_h$ map is well definied]
        \label{lem:beta_h_map_welldef}
        \uses{def:beta_h_map}

        Map $\beta_h$ is well definied.
    \end{lemma}

    \begin{lemma}[$\beta_h$ map is isomorphism of groups]
        \label{lem:beta_h_map_iso}
        \uses{def:beta_h_map}

        Map $\beta_h$ is an isomorphism.
    \end{lemma}

    \begin{lemma}[Fundamental Group doesnt depend on base point]
        \label{fundamental_group_basepoint_dependence}
        \uses{lem:beta_h_map_iso}

        For path connected space fundamental groups based on $x_0, x_1$ are isomorphic. 
    \end{lemma}

    \begin{definition}[Loops in $S^1$]
        \label{def:wn_loop}
        For $n \in \mathbb{Z}$ let us define: 
        $$\omega_n(s) = (\cos{2\pi ns}, \sin{2 \pi ns})$$

        Note: $\omega_n$ is the loop running $n$-times around the circle clockwise or counterclockwise
        depending on the sign of $n$.
    \end{definition}

    \begin{lemma}[$\omega_n$ is a loop]
        \label{lem:wn_loop_is_loop}
        \uses{def:wn_loop, def:loop}

        For each $n \in \mathbb{Z}$: $\omega_n$ is a loop based in $(1,0)$.
    \end{lemma}

    \begin{definition}[Evenly covered]
        \label{def:evenly_covered}

        Let $f : X \to Y$ be a map and $U \subset Y$ be an open set. We say that $U$ is evenly covered
        by $f$ when $f^{-1}(U)$ is a union of disjoint open sets each of which is mapped homeomorphically onto $U$ by $f$.
    \end{definition}

    \begin{definition}[Covering space]
        \label{def:covering_space}
        \uses{def:evenly_covered}

        Given a space $X$, a covering space is a space $\tilde{X}$ together with a map $p : \tilde{X} \to X$ such that:
        For each point $x \in X$, $x$ has a evenly covered neighbourhood $U$ by $p$.
    \end{definition}

    \begin{lemma}[Homotopy lifting property]
        \label{lem:homotop_lifting}
        \uses{def:covering_space}

        Given a map $F:Y\times I \to X$ and a map $\tilde{F}: Y \times \{0\} \to \tilde{X}$ lifting $F|Y\times \{0\}$, then
        there is a unique map $\tilde{F}: Y \times I \to \tilde{X}$ lifting $F$ and restricting to the given $\tilde{F}$ on $Y \times \{0\}$.
    \end{lemma}

    \begin{lemma}[Path lifting property]
        \label{lem:path_lifting}
        \uses{lem:homotop_lifting}

        For each path $f:I \to X$ starting at a point $x_0 \in X$ and each $\tilde{x_0} \in p^{-1}(x_0)$ there
        is a unique lift $\tilde{f}:I \to \tilde{X}$ starting at $\tilde{x_0}$.

    \end{lemma}

    \begin{lemma}[Homotopy path lifting property]
        \label{lem:homotop_path_lifting}
        \uses{lem:homotop_lifting}

        For each homotopy $H : X \times I \to X$ of paths starting at $x_0$ and each $\tilde{x_0} \in p^{-1}(x_0)$ there
        is a unique lifted homotopy $\tilde{H} : X \times I \to \tilde{X}$ of paths starting at $\tilde{x_0}$.
    \end{lemma}

    \begin{lemma}[$\omega_n$ are not homotopic]
        \label{lem:wn_diff_homoclass}
        \uses{lem:homotop_path_lifting, lem:path_lifting, lem:Rn_path_equiv_lemma, lem:wn_loop_is_loop}

        Let $n \neq m$ be integers. Then $\omega_n \nsim \omega_m$.
    \end{lemma}

    \begin{lemma}[All $S^1$ loops are $\omega_n$]
        \label{lem:loop_S1_homotopic_wn}
        \uses{lem:homotop_path_lifting, lem:path_lifting, lem:Rn_path_equiv_lemma, lem:wn_loop_is_loop}

        Let $\gamma$ be a loop in $S^1$ based at $(1,0)$. Then there exists $n \in \mathbb{Z}$
        such that $\gamma \sim \omega_n$.
    \end{lemma}

    \begin{lemma}[Additive structure on $\omega_n$]
        \label{lem:wn_comp_is_sum}
        
        Let $n,m \in \mathbb{Z}$. Then 
        $$[\omega_n] \cdot [\omega_m] = [\omega_{n+m}]$$
    \end{lemma}

    \begin{theorem}[Fundamental group of S1]
        \label{thm:fundamental_group_of_S1}
        \uses{def:fundamental_group,lem:loop_S1_homotopic_wn, lem:wn_diff_homoclass, lem:wn_comp_is_sum}

        $\pi_1(S^1) \cong \mathbb{Z}$
    \end{theorem}
